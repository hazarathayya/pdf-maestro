%doc name should be "`paper_ID.tex"'

\Chapter{RFDMRP:  River Formation Dynamics based Multi-Hop Routing Protocol for Data Collection in Wireless Sensor Networks}{}{}
           
\addcontentsline{toc}{\protect chapter}{\protect\numberline{\hspace{1cm}\it Koppala Guravaiah, R Leela Velusamy, Koppala Guravaiah and R Leela Velusamy}}
\vspace{-2cm}
\begin{center}  
\textit{Koppala~Guravaiah$^{1}$,R.~ Leela~ Velusamy$^{2}$,~Koppala ~Guravaiah$^{1}$,~R.~ Leela~ Velusamy$^{2}$}\\
\textit{$^{1}$Department~ of ~Computer~ Science ~\&~ Engineering,}\\\textit{National Institute of Technology, Tiruchirappalli, India}\\
\textit{$^{2}$Department~ of~ Computer~ Science~ \&~ Engineering,}	\\\textit{National Institute of Technology, Tiruchirappalli, India}
	
\bigskip
\end{center}

\noindent{In Wireless sensor networks, sensor nodes sense the data from environment according to its functionality and forwards to its base station. This process is called Data collection and it is done either direct or multi-hop routing. In
direct routing, every sensor node directly transfers its sensed data to base station which influences the energy consumption from
sensor node due to the far distance between the sensor node and base station. In multi-hop routing, the sensed data is relayed
through multiple nodes to the base station, it uses less energy. This paper introduces a new mechanism for data collection and
routing based on River Formation Dynamics. The proposed algorithm is termed as RFDMRP: River Formation Dynamics based
Multi-hop Routing Protocol. This algorithm is explained and implemented using MATLAB. The performance results are compared with LEACH and MODLEACH. The comparison reveals that the proposed algorithm performs better than LEACH and MODLEACH.}

\noindent\textbf{Keywords:}
Data Collection; Energy efficiency; River Formation Dynamics; Routing Protocols; Wireless Sensor Networks.